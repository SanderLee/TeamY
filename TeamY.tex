\documentclass{article}
\usepackage[utf8]{inputenc}
\usepackage{hyperref}

\title{Kodutöö nr 6}
\author{Sander Leemet, Peeter Kask, Aleksei Lutkov, Mihkel Schleicher,\\ Tomas Kingissepp, Kristiin Kesamaa}
\date{\today}

\begin{document}
\maketitle

\section{Mõelda oma tiimile nimi}
Team Yamaha

\section{Mikko Hypponen - security}
\begin{enumerate}
	\item{Seni kõige suuremad probleemid, mis on seotud Internetiga on privaatsus ja turvalisus.}
	\item{2 Üks suur globaalne probleem on seotud häkkeritega, kes tulevad igalt poolt ja teenivad igal aastal miljoneid, rikkudes seadust ning varastades teiste inimeste tagant.}
	\item{Kui veel 3-4 aasta eest ei tulnud otseselt muretseda terroristide rünnakute eest, siis nüüdseks on olukord muutunud. Terroristid muutuvad järjest paremaks, eriti Islamiriigi tõusmise tõttu ning nüüd on neil reaalselt võimalik rünnata inimesi läbi Interneti.}
	\item{Juba on 1 inimene surnud roboti pärast. Kõik toimus Volkswageni tehases, kus 21 poiss oli valel ajal vales kohas ning jäi robotile ette. Loomulikult ei olnud robot programmeeritud teda tapma.}
	\item{On väga palju inimesi, kes häkivad igale poole sisse lihtsalt sellepärast, et turvalisuse taset tõsta. Nad ei tee midagi halba, kuid otsivad võimalusi, kuidas süsteemi turvalisust parandada.}
	\item{Mikko ütles väga huvitavalt, et talle väga meeldib Google ja tema arust on selle tooted suurepärased, kuid ta sooviks, et ta ei peaks nende toodete kasutamise eest maksma oma andmete ja privaatsusega.}
	\item{Twitter ostab inimeste kohta informatsiooni erinevatest andmebaaside ettevõtetest ning tuvastavad lihtsalt inimese, kes reaalselt Twitterit kasutab, mobiiltelefoni numbri järgi.}
	\item{Mikko arvab ka, et ainuke põhjus, miks Facebook ostis WhatsApp’i ongi seotud sellega, et nad saaksid inimeste mobiiltelefoninumbreid.}
	\item{Enam ei ole häkkimises vaenlased tavalised häkkerid, vaid tegelikult ka valitsus.}
	\item{Suurim vale internetis: ‘’Jah, ma olen lugenud tingimusi ja nõustun nendega.’’}

\section{Chris Rock - death and killing}
Chris Rock räägib selles ettekandes ülemaailmselt levinud (väidetavalt) probleemist, 
kus on võimalik suhteliselt lihtsalt inimesi lihtsalt paberil juriidiliselt "ära tappa" või neid hoopis juurde tekitada. 
Kuigi konkreetseid näiteid tõi ta ainult USA, Kanada ja Austraalia kohta, pidavat see olema levinud väga paljudes riikides 
seoses süsteemide elektroonsesse keskkonda kolimisega. Skeem on päris lihtne, kuna kõike saab veebis teha ja 
arstina surmatunnistuste keskkonda sisselogimiseks vajalik info on avalik (nt nimi, litsentsi number), ning näiteks 
matusekorraldajaks saab ennas samuti interneti vahendusel hõlpsasti teha. Põhjuseid selle tegemiseks võib olla mitmeid: 
näiteks miks mitte nautida iseenda elukindlustusest tulnud raha, olles ise veel eluds? Või hoopis kellegi teise elu väga ebamugavaks 
tegemiseks: nimelt on näiteks kõik ametlikud tegevused ja riikidevaheline reisimine peaaegu võimatu, kui inimene ametlikult surnud on. 
Teiseks on võimalik inimesi lihtsal paberil juurde tekitada, sest sünnitunnistuse koostamine ja registreerimine on isegi lihtsam, 
kui kellegi surnuks tegemine, kuna selleks on vaja ainult arsti või ämmaemandana sisse logida ja andmed sisestada. 
Niimoodi saab näiteks virtuaalse inimesena laene võtta ja mitte kunagi tagasi maksta. Kõige efektiivsem oleks aga ilmselt neid kahte asja koos kasutada: 
näiteks lood uue inimese, teed talle elukindlustuse ja mõne aja pärast "tapad ära". Easy money.\\
TL;DR sünni- ja surmatunnistuste süsteemid on väga ebaturvalised.

\section{Hack all the things. Loetle kõik häkitud seadmed, millest videos juttu.}
\begin{enumerate}
	\item{Epson Artisan 700/800 (Printer)}
	\item{Belkin Wemo (Internet controlled wall plug)}
	\item{Greenwave Reality Smart Bulbs ("Smart" lighting system)}
	\item{File Transporter (Private cloud)}
	\item{Vizio CoStar LT (ISV-B11) (Media player with HDMI Passthrough)}
	\item{Staples Connect (Home automation hub)}
	\item{Amazon FireTV}
	\item{Hisense Android TV (Google TV)}
	\item{LG Smart Refrigerator (LFX31995ST)}
	\item{Vizio Smart TVs (VF55XVT)}
	\item{Sony BDP-S5100 (Blu-Ray Player)}
	\item{LG BP530 (Blu-Ray Player)}
	\item{Panasonic DMP-BDT230 (Blu-Ray Player)}
	\item{Motorola RAZR LTE Baseband}
	\item{PogoPlug Mobile (Cloud Storage Device)}
	\item{Netgear Push2TV (PTV3000)}
	\item{Ooma Telo (Router)}
	\item{Netgear NTV200-100NAS (Media Streaming Device)}
	\item{ASUS Cube (Google TV)}
	\item{Summer Baby Zoom WiFi monitor}
	\item{Samsung SmartCam}
	\item{Wink Hub (Smart home "gateway")}
\end{enumerate}

\section{Ken Westing - stalking. Loetle videos mainitud erinevaid viise, kuidas inimset jälitada.}
\begin{enumerate}
	\item{Isikuandmete saamine (IP aadress,  GPS andmed, aadress, kasutajanimi ja muud) mingi tarkvara kasutades (näiteks USB häkkid, Apple USB Troojalased).}
	\item{Mobiili/arvuti kaamera fotode või videote saamine mingi tarkvara kasutades (USB häkkid, Apple USB Troojalased).}
	\item{Asukoha kindlaksmääramine foto GPS märgiste kasutades.}
	\item{Sotsiaalne jälitamine (facebook, myspace, twitter, flickr).}
	\item{Fotode otsing kaamera seerianumbri kasutades (EXIF otsingumootoris).}
\end{enumerate}

\section{Anna viide ja lühikirjeldus vähemalt 5 huvitavale IT-turvauudisele}
\begin{enumerate}
	\item{\url{https://drownattack.com}\\
		Väga laialt levinud turvaauk. Haavatavad on 33\% kõikidest HTTPS protokolli kasutavatest veebiserveritest. 
		Ründaja saab ligi kogu andmevahetusele kliendi ja serveri vahel.}
	\item{\url{http://www.extremetech.com/mobile/224709-the-gloves-are-off-fbi-argues-it-can-force-apple-to-turn-over-iphone-source-code}\\
		FBI tahab saada ligipääsu Apple IOS lähtekoodile, et teha sinna omale vajalikke muudatusi, mis võivad valmistada probleeme kasutajate privaatsusele.}
	\item{\url{http://www.technewsworld.com/story/83206.html}\\
		Krüptolokker viirused on levinud ka Mac'idele. Krüptolokker viirused krüpteerivad kasutaja failid ning lahti saamiseks tuleb maksta viiruse tootjale teatud summa.}
	\item{\url{http://www.nbcnews.com/tech/tech-news/how-hacker-s-typo-helped-stop-billion-dollar-bank-heist-n536526}\\
		Häkkeri kirjavea tõttu nurjus suurejooneline küberrünnak pangale, mille eesmärk oli sealt raha kanda häkkeri kontodele. Tänu sellele jäi varastamata ligi miljard dollarit, kuid 80 miljonit siiski õnnestus kätte saada.}
	\item{\url{http://www.bbc.com/news/technology-35709676}\\
		Politsei droone on võimalik häkkida 40 dollari suuruse eelarvega. On võimalik võtta droon oma kontrolli alla ning see soovi korral vastu maad lasta.}
\end{enumerate}

\section{Loetle vähemalt 20 viisi, kuidas varastada poest banaan}
\begin{enumerate}
	\item{Peitma tasku.}
	\item{Peitma kotti.}
	\item{Peitma aluspüksidesse.}
	\item{Öelda sõbrale, et ta varastaks.}
	\item{Panna kotti, kus on teised puuviljad.}
	\item{Üks hajutab turvamehe tähelepanu, teine varastab.}
	\item{Kaamerad kummiga kinni kleepida.}
	\item{Elekter välja lülitada.}
	\item{Ütled, et sa oled inspektsioonist ja pead võtma banaani.}
	\item{Sööd banaani koha peal.}
	\item{Kokku leppida turvamehega, et ta ei reageeri.}
	\item{Öelda, et poes on pomm ja kui hakkavad evakueerima, siis varastad.}
	\item{Öelda, et poes on terroristid ja kui hakkavad evakueerima, siis varastad.}
	\item{Kasutada inimeste vastast keemiarelva ja peale seda varastada.}
	\item{Teha mitu vargust, kui turvamehed hakkavad esimesega varusega tegelema, siis varastad.}
	\item{Värvida seda ja siis see ei ole juba banaan.}
	\item{Võtta relv turvamehelt ja varastada banaan.}
	\item{Panna kätte poksi kindad, turvamehed ei hakka siis kaklema.}
	\item{Ventilatsioon abiga varastada.}
	\item{Sõita poodi sisse monorattaga ja varastada, turvamehed ei jõua reageerida.}
\end{enumerate}

\section{Loetle vähemalt 20 viisi, kuidas takistada poest banaanide vargust}
\begin{enumerate}
	\item{Mitte müüa banaane, siis ei varastatakski.}
	\item{Panna banaanide asemel midagi nende sarnast (petta varast).}
	\item{Banaane saab ainult otse müüjalt (kassast) osta.}
	\item{Turvamees banaane valma panna.}
	\item{Panna banaanidel ümber mingi turvaelement, mis hakkaks piiksuma kui ilma maksmata nendega välja minnakse.}
	\item{Panna kaitseks ümber seisma sõdurid.}
	\item{Panna mingi hoiatav või hirmutav silt.}
	\item{Minimeerida banaanide ümbrus.}
	\item{Snaiper vales, kui keegi varastab siis on luba tulistada.}
	\item{Panna ümber koerad või mõni muu ohtlik loom, kes ründavad ainult varast.}
	\item{Ahvid poodi valvesse.}
	\item{Kõigepealt tuleb maksta siis saad banaanid.}
	\item{Banaane saab osta ainult masinast, kui sisestada raha, siis annab masin banaanid.}
	\item{Poest saab esitada ainult tellimusi, mingi inimene või robot paneb soovitud ostu kokku ja ostja peab vaid maksma.}
	\item{Tellimust saab esitada ainult interneti teel ja kaup tuuakse koju kätte.}
	\item{Väljas inimestele on näha ainult halvaks läinud banaanid, aga kassas makstes saab need uute vastu vahetada.}
	\item{Banaanid on suletud ruumis ja kätte saab ainult ID-kaardi põhiselt.}
	\item{Poodi sisenemiseks ja väljumiseks on vaja esitada isikuttõendavdokument.}
	\item{Poodi sisenemiseks on vaja maksta nt. 10 eurot ja pärast lahkudes, näidates tsekki saab raha tagasi.}
	\item{Igale ostule, mis on suurem kui 10 eurot antakse kaasa 4 tasuta banaani, väiksematele summadele vähem kui 4 banaani.}
\end{enumerate}

\section{Loetle vähemalt 3 häkkimisprojekti by Samy Kamkar}
\begin{enumerate}
	\item{MagSpoof - Seadeldis millega saab emuleerida magnetkaarte. Link: \url{http://samy.pl/magspoof/}}
	\item{KeySweeper - Arduinol põhinev seade juhtmevabade klaviatuuride \textit{sniffimi}'seks. Link: \url{http://samy.pl/keysweeper/}}
	\item{SkyJack - Raspberry PI'ga droon, mis on võimeline juhtmevabalt kaaperdama teisi droone. Link: \url{http://samy.pl/skyjack/}}
\end{enumerate}
\end{document}
