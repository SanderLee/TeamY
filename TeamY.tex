\documentclass{article}
\usepackage[utf8]{inputenc}
\usepackage{hyperref}

\title{Kodutöö nr 6}
\author{Sander Leemet, Peeter Kask, ...}
\date{\today}

\begin{document}
\maketitle

\section{Mõelda oma tiimile nimi}
Team Yamaha?

\section{Mikko Hypponen - security}
10 mõtet video kohta.

\section{Chris Rock - death and killing}
Sisukokkuvõte 10..20 lauset.

\section{Hack all the things}
Loetle kõik häkitud seadmed, millest videos juttu.

\section{Ken Westing - stalking}
Loetle videos mainitud erinevaid viise, kuidas inimset jälitada.

\section{Anna viide ja lühikirjeldus vähemalt 5 huvitavale IT-turvauudisele}
\begin{enumerate}
	\item{\url{https://drownattack.com}\\
		Väga laialt levinud turvaauk. Haavatavad on 33\% kõikidest HTTPS protokolli kasutavatest veebiserveritest. 
		Ründaja saab ligi kogu andmevahetusele kliendi ja serveri vahel.}
	\item{...}
	\item{...}
\end{enumerate}

\section{Loetle vähemalt 20 viisi, kuidas varastada poest banaan}
\begin{enumerate}
	\item{Paned tasku kui keegi ei vaata}
	\item{Paned tasku ja jooksed turvamehe eest ära}
	\item{...}
\end{enumerate}

\section{Loetle vähemalt 20 viisi, kuidas takistada poest banaanide vargust}
\begin{enumerate}
	\item{...}
	\item{...}
	\item{...}
\end{enumerate}

\section{Loetle vähemalt 3 häkkimisprojekti by Samy Kamkar}
\begin{enumerate}
	\item{MagSpoof - Seadeldis millega saab emuleerida magnetkaarte. Link: \url{http://samy.pl/magspoof/}}
	\item{KeySweeper - Arduinol põhinev seade juhtmevabade klaviatuuride \textit{sniffimi}'seks. Link: \url{http://samy.pl/keysweeper/}}
	\item{SkyJack - Raspberry PI'ga droon, mis on võimeline juhtmevabalt kaaperdama teisi droone. Link: \url{http://samy.pl/skyjack/}}
\end{enumerate}
\end{document}